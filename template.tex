%%%%%%%%%%%%%%%%%%%%%%%%%%%%%%%%%%%%%%%%%
% My personal Resume/CV
% LaTeX Template
% Version 1.0 (2022-02-17)
%
% This template has been downloaded from:
% http://www.LaTeXTemplates.com
%
% Original author:
% Carmine Spagnuolo (cspagnuolo@unisa.it) with major modifications by 
% Vel (vel@LaTeXTemplates.com)
%
% License:
% The MIT License (see included LICENSE file)
%
%%%%%%%%%%%%%%%%%%%%%%%%%%%%%%%%%%%%%%%%%

%----------------------------------------------------------------------------------------
%	PACKAGES AND OTHER DOCUMENT CONFIGURATIONS
%----------------------------------------------------------------------------------------
\documentclass[letterpaper]{hosmacv} % a4paper for A4

%----------------------------------------------------------------------------------------
%	 PERSONAL INFORMATION
%----------------------------------------------------------------------------------------

% If you don't need one or more of the below, just remove the content leaving the command, e.g. \cvnumberphone{}
\cvname{Hossein Mani} % Your name

\cvjobtitle{Electrical engineer \\ Digital communication \\ Cryptography } % Job title/career

\personalinfo{Personal information}

\cvlinkedin{https://dk.linkedin.com/in/hossein-mani-orig}
\cvnumberphone{+45 42 72 59 62} % Phone number
\cvmail{hosma@dtu.dk} % Email address
\cvsite{https://github.com/mani8785} % Personal website
\cvhome{Engbakken 43, 2830 Virum} % Personal website
\cvlocation{Denmark}

\cvmailreffirst{ulrik.andersen@fysik.dtu.dk}
\cvmailrefsec{tobias.gehring@fysik.dtu.dk}
\cvmailrefthird{Christoph.Pacher@ait.ac.at}
\cvmailrefforth{shemati@uidaho.edu}
\cvmailreffifth{hsaeedi@modares.ac.ir}


%----------------------------------------------------------------------------------------

\begin{document}

\makeprofile % Print the sidebar for page 1

%----------------------------------------------------------------------------------------
%	 EXPERIENCE
%----------------------------------------------------------------------------------------
\profilesection{Work Experience\\}{11.0cm} 

\begin{hosmaenv} % Environment for a list with descriptions
	\hosmaitem
    	{2020 - now}
        {Postdoctoral researcher}
        {\href{https://www.dtu.dk/english}{Technical University of Denmark (DTU)}, Lyngby, Denmark}
        {\href{https://www.bigq.fysik.dtu.dk/about-bigq}{Center for Macroscopic Quantum States}}
        {\begin{itemize}\itemsep -2pt
            \item Digital communication and error correction specialist.
            \item Information theory and cryptography specialist.
            \item Developing post-processing software for quantum key distribution(QKD).
        \end{itemize} }
	%\twentyitem{<dates>}{<title>}{<organization>}{<location>}{<description>}
	
	\hosmaitem
    	{2017 - 2019}
        {Researcher \& Software engineer}
        {\href{https://www.ait.ac.at/en/}{Austrian Institute of Technology (AIT)}, Vienna, Austria}
        {\href{https://www.ait.ac.at/en/solutions/cyber-security}{Section of Cyber-Security}}
        {\begin{itemize}\itemsep -2pt
            \item Error correction specialist.
            \item Highly efficient code design for quantum communication system.
            \item Developing high throughput GPU based signal processing software.
        \end{itemize} }
    
    \hosmaitem
    	{2015 - 2017}
        {Researcher \& software engineer}
        {\href{https://www.uidaho.edu/}{University of Idaho}, Idaho, United States}
        {\href{https://www.uidaho.edu/engr/departments/ece}{Department of  Electrical and Computer Engineering}}
        {\begin{itemize}\itemsep -2pt
            \item Physical layer security in wireless communication.
            \item Developing device independent telecommunication blocks.
            \item Low power circuit design for communications systems
            \item Stochastic decoder design for telecommunication systems.
            \item Teaching and student mentoring. 
        \end{itemize} }
        
    \hosmaitem
    	{2012 - 2015}
        {Researcher \& software engineer}
        {\href{https://www.modares.ac.ir/en-ece}{Wireless innovation Laboratory}, Tehran, Iran}
        {{Department of  Electrical and Computer Engineering}}
        {\begin{itemize} \itemsep -2pt % Reduce space between items
            \item National broadband cellular network planning. (3G/LTE).
        \end{itemize}}
        
	
\end{hosmaenv}


%----------------------------------------------------------------------------------------
%	 EDUCATION
%----------------------------------------------------------------------------------------
\profilesection{Education\\}{11.0cm} 

\begin{hosmaenv} % Environment for a list with descriptions
	\hosmaitem
    	{2017 - 2020}
        {PhD in Electrical Engineering/Quantum Physics}
        {\href{https://www.dtu.dk/english}{Technical University of Denmark (DTU)}, Lyngby, Denmark}
        {\href{https://www.bigq.fysik.dtu.dk/about-bigq}{Center for Macroscopic Quantum States}}
        {\begin{itemize}
            \item Information Reconciliation protocols for quantum key distribution.
        \end{itemize}}
	%\twentyitem{<dates>}{<title>}{<organization>}{<location>}{<description>}
	
	\hosmaitem
    	{2009 - 2012}
        {Master of Science in Electrical Engineering-Telecommunication}
        {\href{https://modares.ac.ir/en}{Tarbiat Modares University (TMU)}, Tehran, Iran}
        {\href{https://www.modares.ac.ir/en-ece}{Department of Electrical \& Computer Engineering}}
        {\begin{itemize}
            \item Cumulative GPA: 16.98/20
        \end{itemize}}
    
    \hosmaitem
    	{2005 - 2009}
        {Bachelor of Science in Electrical Engineering-Electronics}
        {\href{http://www.ufabc.edu.br/}{Shahid Beheshti University (SBU)}, Tehran, Iran}
        {\href{http://en.sbu.ac.ir/Faculties/Electrical/Pages/default.aspx}{Department of Electrical Engineering}}
        {\begin{itemize}
            \item Cumulative GPA: 17.11/20
        \end{itemize}}
	
\end{hosmaenv}


%----------------------------------------------------------------------------------------
%	 AWARD
%----------------------------------------------------------------------------------------

\profilesection{Awards\\}{11.0cm} 

\begin{hosmashort}
\hosmaitemshort{IEEE best presentation award}{University of Idaho, USA.}{2017}
\hosmaitemshort{Travel award for IEEE rising stars conference}{Las Vegas, USA.}{2016}
\hosmaitemshort{Awarded scholarship for master program}{Tarbiat Modares University, Tehran, Iran.}{2009}
\end{hosmashort}

% \begin{hosmaenv}
%     \hosmaitem{2017}{IEEE best presentation award}{University of Idaho, USA.}{}{}
%     \hosmaitem{2016}{Travel award for IEEE rising stars conference}{Las Vegas, USA.}{}{}
%     \hosmaitem{2009}{Awarded scholarship for master program}{Tarbiat Modares University, Tehran, Iran.}{}{}
% \end{hosmaenv}

\clearpage

\makeprofilesecondpage % Print the sidebar


%----------------------------------------------------------------------------------------
%	 Sellected Publications
%----------------------------------------------------------------------------------------
\profilesection{Selected Publications\\}{11.0cm} 

\begin{hosmashort}
\hosmaitemshort{Ph.D.  dissertation.}
{Information reconciliation  protocols  for  CV-QKD.}
{2020}

\hosmaitemshort{Book.}
{Digital design. (Persian context)}
{2012}

\hosmaitemshort{Peer review journals.}
{
\begin{itemize} % Reduce space between items
    \item \textbf{H. Mani}, et.al. ``Multiedge-type   low-density   parity-checkcodes   for   continuous-variable   quantum   key   distribution'', Phys. Rev. A. vol. 103, p. 062419 \textbf{2021}.
    \item N. Jain, H. Chin, \textbf{H. Mani}, et.al. ``Practical continuous-variable quantum key distribution with composable security'', 	arXiv:2110.09262 2021.
    % \item \textbf{H. Mani}, et.al. ``Multi-edge-type LDPC code design with G-EXIT charts for continuous-variable quantum key distribution'', arXiv:1812.05867v1 2019.
    \item \textbf{H. Mani} and S. Hemati. ``Symbol-level Stochastic Chase Decoding of Reed-Solomon and BCH Codes'', IEEE Transaction On Communications, 2019.
    \item \textbf{H. Mani}, H. Saeedi, ``Message Passing-Based Decoding of Convolutional Codes: Performance and Complexity Analysis,'' in Communications Letters, IEEE , vol.20, no.2, pp.216-219, Feb. 2016.
\end{itemize}
}{}\\

\hosmaitemshort{Conferences}
{\begin{itemize}
    \item \textbf{H. Mani}, et.al.
``{An Approximation Method for Analysis and Design of Multi-Edge Type LDPC Codes}'', Qcrypt \textbf{2018}.
\item \textbf{H. Mani}, et.al, ``{Resource allocation based on the message
passing algorithm in underlay cognitive networks}'', WCNC \textbf{2014}.
\item \textbf{H. Mani1}, et.al, ``{On Generalized EXIT Charts of LDPC Code
Ensembles over Binary-Input Output- Symmetric Memory less Channels}'', ISIT \textbf{2012}.
\item \textbf{H. Mani}, H. Saeedi,``{ \textit{Generalized EXIT Charts for Irregular LDPC codes}}'', AISP \textbf{2011}.
\end{itemize}}
{}

\end{hosmashort}

% ==============================================================
% \profilesection{Reviewer\\}{11.0cm}

% \begin{hosmashort}
% \hosmaitemshort{Some of the selected journals}
% {\begin{itemize}
%     \item IEEE Wireless Communications Letters, 
%     \item IEEE Wireless Communications, 
%     \item IEEE transaction of information theory,
%     \item IEEE Journal On Selected Areas In Communications, 
%     \item IEEE Transactions on Information Forensics and Security. 
% \end{itemize}
% }
% {2015-now}
% \end{hosmashort}

% ==============================================================
\profilesection{References\\}{11.0cm}

\cvdoublecolumn{\cvreference{Prof. Ulrik Lund Andersen}
    {Technical University of Denmark}
    {Bld. 307, room 254}
    {2800 Kgs. Lyngby, Denmark}
    {}
    {\href{mailto:\cvmailreffirst}{\cvmailreffirst}}
    {}%
    }
    {\cvreference{Assoc.Prof. Tobias Gehring}
    {Technical University of Denmark}
    {Bld. 307, room 262}
    {2800 Kgs. Lyngby, Denmark}
    {}
    {\href{mailto:\cvmailrefsec}{\cvmailrefsec}}
    {}%
    }

\cvdoublecolumn{\cvreference{Dr. Christoph Pacher}
    {Center of Digital Safety \& Security}
    {Austrian Institute of Technology}
    {Giefinggasse 4, 1210 Vienna}
    {}
    {\href{mailto:\cvmailrefthird}{\cvmailrefthird}}
    {}
    }
    {\cvreference{Prof. Saied Hemati}
    {NVM Circuits and Algorithms}
    {Intel Corporation}
    {Folsom, California CA}
    {}
    {\href{mailto:\cvmailrefforth}{\cvmailrefforth}}
    {}
    }
    
% \begin{hosmashort}
% \hosmaitemshort{GPSA (Graduate \& Professional Student Association), Idaho, USA}
% {Senator at large}
% {2016-2017}
% \hosmaitemshort{Iranian Student Association, Idaho, USA}
% {Vice-president}
% {2016-2017}
% \hosmaitemshort{IEEE (Institute of Electrical and Electronics Engineers)}
% {member}
% {2009-now}
% \end{hosmashort}

% \nocite{mani2021poster, PhysRevA.103.062419, thesis_mani, mani2019recWC, mani2012onGEXIT, mani2020two-code, mani2019AnAM, mani2016mp, mani2019AlgorithmicAT, mani2019rs,  mani2014ra, mani2011GEXIT}




% \bibliographystyle{alpha}
% \bibliography{Bibliography}



% \begin{hosmaenv}
%     \hosmaitem{2020}{Ph.D.  dissertation.}{}{H. Mani}{}
%     \hosmaitem{2012}{Book/Book Chapter}{Digital Design.}{H. Mani, A. Fatemi, ISBN: 978-964-7668-65-1}{}
%     \hosmaitem{}{Selected Journals}{}{}{\begin{itemize} \itemsep 4pt % Reduce space between items
% \item \textbf{H. Mani}, et.al.
% ``{Multiedge-type   low-density   parity-checkcodes   for   continuous-variable   quantum   key   distribution}'', Phys. Rev. A. vol. 103, p. 062419 \textbf{2021}.
% \item N. Jain, H. Chin, \textbf{H. Mani}, et.al.
% ``{Practical continuous-variable quantum key distribution with composable security}'', 	arXiv:2110.09262 2021.
% \item \textbf{H. Mani}, et.al.
% ``{Multi-edge-type LDPC code design with G-EXIT charts for continuous-variable quantum key distribution}'', arXiv:1812.05867v1 2019.
% \item \textbf{H. Mani} and S. Hemati.
% ``{Symbol-level Stochastic Chase Decoding of Reed-Solomon and BCH Codes}'', IEEE Transaction On Communications, 2019. % Reduce space between items
% \item \textbf{H. Mani}, H. Saeedi, ``{Message Passing-Based Decoding of Convolutional Codes: Performance and
% Complexity Analysis},'' in Communications Letters, IEEE , vol.20, no.2, pp.216-219, Feb. 2016.
% \end{itemize}}
% \hosmaitem{}{Selected Conferences}{}{}{\begin{itemize} \itemsep 4pt % Reduce space between items
% \item \textbf{H. Mani}, et.al.
% ``{An Approximation Method for Analysis and Design of Multi-Edge Type LDPC Codes}'', Qcrypt 2018.
% \item \textbf{H. Mani}, et.al.
% ``{An Approximation Method for Analysis and Design of Multi-Edge Type LDPC Codes}'', Qcrypt 2018.
% \item \textbf{H. Mani}, et.al, ``{Resource allocation based on the message
% passing algorithm in underlay cognitive networks}'', WCNC 2014.
% \item \textbf{H. Mani}, et.al, ``{On Generalized EXIT Charts of LDPC Code
% Ensembles over Binary-Input Output- Symmetric Memory less Channels}'', ISIT 2012.
% \item \textbf{H. Mani}, H. Saeedi,``{ \textit{Generalized EXIT Charts for Irregular LDPC codes}}'', AISP 2011.
% \end{itemize}}
% \end{hosmaenv}


% \begin{twenty} % Environment for a list with descriptions
% 	\twentyitem
%     	{2010 - 2017 \\ (Esperado)}
%         {PhD}
%         {\href{http://www.ufabc.edu.br/}{Technical University of Denmark (DTU)}}
%         {Santo André, São Paulo}
%         {Progresso Atual: 94\% \\ \textit{Trabalho de Conclusão: Redes Definidas por Software}}
% 	\twentyitem
%     	{2010 - 2013}
%         {Bacharelado em Ciência e Tecnologia}
%         {\href{http://www.ufabc.edu.br/}{Universidade Federal do ABC}}
%         {Santo André, São Paulo}
%         {}
% 	%\twentyitem{<dates>}{<title>}{<organization>}{<location>}{<description>}
% \end{twenty}


% \section{Research}
% \begin{twenty}
% 	\twentyitem
%     	{2015 - 2017}
%         {Engenharia de Informação}
%         {\href{http://www.ufabc.edu.br/}{Universidade Federal do ABC}}
%         {}
%         {
%         {\begin{itemize}
%         \item Investigação da evolução dos sistemas de rede e sua dinâmica.
%         \item Proposta para as redes atuais para que seja possível suprir seu crescimento gerando poucos impactos na estrutura existente com o uso do conceito de redes definidas por software 
%         \item Utilização de SDN-WIFI para simulação de ambientes de mobilidade.
       
%     \end{itemize}}
%         }
% \end{twenty}

\end{document} 
